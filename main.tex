% !TeX TS-program = xelatex

\documentclass{resume}
\ResumeName{冯开宇}

\begin{document}

\ResumeContacts{
  (+86) 188-8888-8888,%
  \ResumeUrl{mailto:loveress01@outlook.com}{loveress01@outlook.com},%
  \ResumeUrl{https://blog.fkynjyq.com}{blog.fkynjyq.com} \footnote{下划线内容包含超链接。},%
  \ResumeUrl{https://github.com/fky2015}{github.com/fky2015}%
}

\ResumeTitle

\section{个人总结}

\begin{itemize}
  \item 本人乐观开朗、在校成绩优异、自驱能力强,具有良好的沟通能力和团队合作精神,可以使用英语进行工作交流(六级成绩 xxx)。
  \item 有六年 Linux 使用经验,较为丰富的软件开发经验、开源项目贡献和维护经验。善于技术写作,持续关注互联网技术发展。
  \item 求职意向:分布式系统开发及相关实习工作。
\end{itemize}

\section{教育经历}
\ResumeItem
[北京理工大学大学|硕士研究生]
{北京理工大学}
[\textnormal{网络空间安全,网络空间安全学院|}  学术型硕士研究生]
[2021.09—2024.06(预计)]

主要研究方向为\textbf{拜占庭共识算法},在区块链和分布式系统领域方面有一定的研究和工程经验。\textbf{2024年应届生}。

主要研究成果为:发表在 XXX 期刊上的论文《XXX》。

\ResumeItem
[北京理工大学大学|本科生]
{北京理工大学}
[\textnormal{计算机科学与技术,计算机学院|} 工学学士]
[2017.09—2021.06]

\textbf{GPA: 3.7/4.0(专业前 5\%)},获学业奖学金多次,全国大学生XYZ竞赛二等奖,XXX竞赛三等奖。


\section[技术能力]{技术能力\protect\footnote{与求职岗位无关的技能省略或用灰色表示。}}
\begin{itemize}
  \item \textbf{前后端}: Rust, Golang, Python, MySQL, \GrayText{React, Vue.js, JavaScript/TypeScript, HTML/CSS}
  \item \textbf{运维}: Linux, Docker, Shell, Kubernetes, \GrayText{Ansible}.
  \item \textbf{关键词}: Tokio.rs, Consensus Algorithm, Vim, Git, Kafka, Markdown, \GrayText{Next.js}, \GrayText{\LaTeX}, \GrayText{Django}
\end{itemize}

\section{工作经历}

\ResumeItem{北京ABCD有限公司}
[后端开发实习生/XXXX]
[2020.10—2021.03] 

\begin{itemize}
  \item \textbf{独立负责XXX业务后端的设计、开发、测试和部署。}通过 FaaS、Kafka 等平台实现站内信模板渲染服务。向上游提供 SDK 代码,增加或升级了多种离线和在线逻辑。
  \item \textbf{参与XXX的需求分析,系统技术方案设计;完成需求开发、灰度测试、上线和监控。}
\end{itemize}

\section{项目经历}

\ResumeItem{\textbf{ZYX} 平台下的某某共识算法设计与实现}
[共识算法设计与实现]
[2021.11—2022.07] 

\begin{itemize}
  \item 根据 ZYX (Rust 实现的开源区块链框架) 的架构,\textbf{修改并实现某某某共识算法}。
  \item 针对系统进行性能测试,分析瓶颈,并优化吞吐量;TPS 由 1K 达到 6K。
  \item 此项目为实验室研究项目的一部分。
\end{itemize}

\ResumeItem[BIThesis 北京理工大学毕设模板集合(开源项目)]
{\ResumeUrl{https://github.com/BITNP/BIThesis}{\textbf{BIThesis} 北京理工大学毕设 \LaTeX 模板集合(开源项目)}}
[主要维护者]
[2020.04 至今]

\begin{itemize}
  \item 根据相关排版要求,\textbf{利用 LaTeX3 (expl3) 设计了同时符合各个学位要求且支持灵活配置的宏包及多套模板}。
  \item 需求开发和问题修复采用标准工作流,引入了回归测试与基于 GitHub Actions 的测试与持续集成。
  \item 负责了什么什么;完成了怎样的结果。
\end{itemize}


\ResumeItem{区块链在线交互教学实验平台}
[架构设计、全栈开发]
[2020.12—2021.06] 

\begin{itemize}
  \item 设计并实现了\textbf{基于 Kubernetes 的在线教学平台架构,支持基于用户或者班级的动态容器编排}。
  \item 设计并实现\textbf{一套前后端的平台框架代码}(前端采用 Vue.js,后端采用 Spring Boot),提供路由、数据统计、用户信息管理以及 K8s 实验环境管理能力,使团队内的其他同学只需要专注于开发具体交互教学实验逻辑。
  \item 基于此平台实现了基于 Hyperledger Fabric 的交互教学实验。
  \item 编写了相关的 CI/CD 以及部署脚本,在多个环境下部署(脚本或 Ansible)并运维此系统。该系统应用于在多位老师的教学工作和研究工作中。
\end{itemize}

\ResumeItem{电脑诊所线上预约系统}
[架构设计、全栈开发]
[2019.07—2019.10] 

\begin{itemize}
  \item 需求源自于我加入的学生组织日常开展的电脑义诊活动。
  \item 后端采用 Django + REST API;前端(包括管理后台和用户应用)采用 Vue.js 基于 Vuetify/VUX 实现 SPA,开发采用 GitLab 及其 CI (GitLab Runner),运维使用 Docker Compose 管理容器。
  \item 完成需求分析、系统设计和代码实现。与学校积极对接,接入了学校的统一服务平台,向全校师生提供服务。
  \item 该系统目前仍在投入使用。
\end{itemize}

\end{document}
