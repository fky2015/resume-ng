\documentclass{resume}

\usepackage[colorlinks=false,hidelinks,
pdfborderstyle={/S/U/W 1},urlbordercolor={0 0 0},
]{hyperref}
\usepackage{xcolor}
\usepackage{lipsum}

\ResumeName{冯开宇}

\begin{document}

\ResumeContacts{
  (+86) 188-8888-8888,%
  \href{mailto:loveress01@outlook.com}{loveress01@outlook.com},%
  \href{https://blog.fkynjyq.com}{https://blog.fkynjyq.com} \footnote{下划线内容包含超链接。},%
  % \href{https://github.com/fky2015}{https://github.com/fky2015}%
}

\ResumeTitle

\section{个人总结}

\begin{itemize}
  \item 本人
\end{itemize}

有开源项目维护经验。

\section{教育经历}
\ResumeItem{{\heiti 北京理工大学},网络空间安全|网络空间安全学院,\textit{学术型硕士研究生}}[2021.09 - 2024.06(预计)]
主要研究方向为\textbf{拜占庭共识算法},在区块链和分布式系统领域方面有一定的研究和工程经验。
\ResumeItem{{\heiti 北京理工大学},计算机科学与技术|计算机学院,\textit{工学学士}}[2017.09 - 2021.06]
\textbf{排名5/xxx(前5\%)},获学业奖学金多次,全国大学生信息安全竞赛二等奖(2次),龙芯杯三等奖。


\section{技术能力}
\begin{itemize}
  \item \textbf{后端}: Rust, Golang, Python.
  % \item \textbf{前端}: React, Vue.js, TypeScript, HTML/CSS.
  \item \textbf{运维}: Docker, Kubernetes, Linux.
  \item \textbf{关键词}: 
\end{itemize}

\section{工作经历}

\ResumeItem{{\heiti 北京字节跳动科技有限公司}|\textit{后端开发实习生|番茄小说}}[2020.10 - 2021.03] 
\begin{itemize}
  \item 独立负责站内信业务的设计、开发、测试和部署。通过 FaaS、Kafka 实现站内信模板渲染服务,解决重发漏发问题。向上游提供 SDK 代码,增加或升级了多种离线和在线的发送逻辑。
  \item 参与作家后台、用户后端的需求分析,设计系统技术方案,完成开发、灰度测试、上线和监控。
\end{itemize}


\section{项目经历}

\ResumeItem{{\heiti Substrate平台下的拜占庭共识算法实现} | \textit{共识算法设计与实现}}
[2021.11 - 2022.07] 
\begin{itemize}
  % TODO: 
  \item 根据 Substrate (Rust 实现的开源区块链框架) 的架构设计,修改并实现 PBFT 和 Tendermint 的共识算法。
  \item 研究底层代码,为团队内的其他同学设计智能合约、应用层代码的架构提供技术支持。
  \item 针对系统进行性能测试,并优化吞吐量。 
\end{itemize}

\ResumeItem{{\heiti \href{https://github.com/BITNP/BIThesis}{BIThesis 北京理工大学毕设 LaTeX 模板集合(开源项目)}}  | \textit{主要维护者}}
[2020.04 至今]
\begin{itemize}
  \item 根据相关排版要求,利用 LaTeX3 (expl3) 设计一套符合要求且支持灵活配置的毕设模板。
  \item 需求开发和问题修复采用标准工作流,利用 Github Actions 进行多种测试。模板代码和使用手册分发至多个平台。
  \item 负责积极与学校进行沟通,获得了学校的认可和支持;维护社区,及时响应用户的需求;并撰写了大量文档。目前模板下载量超过4000次。
\end{itemize}


\ResumeItem{{\heiti 区块链在线教学实验平台} | \textit{架构设计、全栈开发}}
[2020.12 - 2021.06] 

\begin{itemize}
  \item 设计并实现了基于 Kubernetes 的在线教学平台架构,支持基于用户或者班级的动态容器编排。
  \item 设计并实现一套前后端的平台框架代码(前端采用 Vue.js,后端采用 Spring Boot),提供路由、数据统计、用户信息管理以及实验环境管理能力,使团队内的其他同学只需要专注于开发具体教学实验逻辑。
  \item 基于此平台,实现了基于 Hyperledger Fabric 的交互教学实验,并协助其他同学实现了比特币、以太坊的相关实验。
  \item 编写了相关的 CI/CD 以及部署脚本,在多个环境下部署并运维此系统。
\end{itemize}

\ResumeItem{{\heiti 电脑诊所线上预约系统} | \textit{架构设计、全栈开发}}
[2019.07 - 2019.10] 

\begin{itemize}
  \item 后端采用 Django + REST API;前端(包括管理后台和用户应用)采用 Vue.js 基于 Vuetify/VUX 实现 SPA,开发采用 GitLab 及其 CI,运维使用 Docker-compose 组织容器。
  \item 完成需求分析、系统设计和代码实现。与学校积极对接,接入了学校的统一服务平台,向全校师生提供服务。该系统目前仍在投入使用。
\end{itemize}

\end{document}
