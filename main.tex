\documentclass{resume}
\ResumeName{冯开宇}

\begin{document}

\ResumeContacts{
  (+86) 188-8888-8888,%
  \href{mailto:loveress01@outlook.com}{loveress01@outlook.com},%
  \href{https://blog.fkynjyq.com}{blog.fkynjyq.com} \footnote{下划线内容包含超链接。},%
  \href{https://github.com/fky2015}{github.com/fky2015}%
}

\ResumeTitle

\section{个人总结}

\begin{itemize}
  \item 本人乐观开朗、在校成绩优异、自驱能力强,具有良好的沟通能力和团队合作精神,可以使用英语进行工作交流。
  \item 有六年 Linux 使用经验,较为丰富的软件开发经验、开源项目贡献和维护经验,持续关注互联网技术发展。
  \item 求职意向:分布式系统开发。
\end{itemize}

\section{教育经历}
\ResumeItem
[北京理工大学大学|硕士研究生]
{北京理工大学}
[\textnormal{网络空间安全,网络空间安全学院|}  学术型硕士研究生]
[2021.09 - 2024.06(预计)]

主要研究方向为\textbf{拜占庭共识算法},在区块链和分布式系统领域方面有一定的研究和工程经验。\textbf{2024年应届生}。

\ResumeItem
[北京理工大学大学|本科生]
{北京理工大学}
[\textnormal{计算机科学与技术,计算机学院|} 工学学士]
[2017.09 - 2021.06]

\textbf{排名5/xxx(前5\%)},获学业奖学金多次,全国大学生信息安全竞赛二等奖(2次),龙芯杯三等奖。


\section[技术能力]{技术能力\protect\footnote{与求职岗位无关的技能省略或用灰色表示。}}
\begin{itemize}
  \item \textbf{后端}: Rust, Golang, Python.
  % \item \textbf{前端}: React, Vue.js, TypeScript, HTML/CSS.
  \item \textbf{运维}: Docker, Kubernetes, Linux.
% 10s内能够让面试官看上的简历,肯定要有热门的技术词汇,最好还是和面试官(他们也是工程师)息息相关的技术。
% 这一点大家经常有个误区,觉得我一定要高大上,简历上的技术越新越好。但是,这些技术往往还没来得及大规模推广,甚至面试官都不知道,很可能写上去是没有用的。反之有些技术,比如Test、Log、Git这些,看上去并不fancy,但是这是每一个工程师天天要做的工作,反而能把你和其他new grad区分开来。不要小瞧这一点,包括一些CS的master很多都不会Test、不会分析Log、不会Git。
%
% 你的简历如果既有一些较新的技术(一般出来一两年左右的,类似Go、React),又有经典的必备的技能,那么就一定能够吸引他的眼球。
  \item \textbf{关键词}: Vim, Git
\end{itemize}

\section{工作经历}

\ResumeItem{北京字节跳动科技有限公司}[后端开发实习生/番茄小说][2020.10 - 2021.03] 
\begin{itemize}
  \item 独立负责站内信业务的设计、开发、测试和部署。通过 FaaS、Kafka 等平台实现站内信模板渲染服务。向上游提供 SDK 代码,增加或升级了多种离线和在线的发送逻辑。
  \item 参与作家后台、用户后端的需求分析,设计系统技术方案,完成开发、灰度测试、上线和监控。
\end{itemize}


% 还有一些同学,尤其是CS Master和转专业的同学,有这样的疑问,如何让我的简历显得更有深度?我认为你的简历需要体现设计和实现上的复杂性。我总结了一个描述的技巧:起承转合。
%
% 第一行,起。写清楚项目的背景。写一下研究过什么同类的产品,我的产品的优势是什么。这能告诉面试官我不是随意设计一个项目的,是有目的、有规划的。
%
% 第二行,承。一般我会写基本的实现。用了什么框架、什么技术。记得要把context交代清楚。
%
% 第三行,转。描述遇到的挑战,是如何解决的。通过这条,说明我这个项目不是应付交差,而是做了一段时间,遇到了问题,并且解决了问题。
%
% 第四行,合。描述最终的结果。我是如何delivere、present、test这个项目的。告诉面试官我有ownership,能保证产品的最终完成。最好可以用一些数字来体现结果,而不是空洞的描述。

\section{项目经历}

\ResumeItem{Substrate平台下的拜占庭共识算法实现}
[共识算法设计与实现]
[2021.11 - 2022.07] 

\begin{itemize}
  % TODO: 
  \item 根据 Substrate (Rust 实现的开源区块链框架) 的独特架构设计,修改并实现 PBFT 和 Tendermint 的共识算法。
  % TODO: maybe remove this
  \item 研究底层代码,为团队内的其他同学设计智能合约、应用层代码的架构提供技术支持。
  \item 针对系统进行性能测试,并优化吞吐量。 
\end{itemize}

\ResumeItem{\href{https://github.com/BITNP/BIThesis}{BIThesis 北京理工大学毕设 LaTeX 模板集合(开源项目)}}
[主要维护者]
[2020.04 至今]

\begin{itemize}
  \item 根据相关排版要求,利用 LaTeX3 (expl3) 设计了一套同时符合各个学位要求且支持灵活配置的毕设模板及宏包。
  \item 需求开发和问题修复采用标准工作流,利用 Github Actions 进行多种测试。模板代码和使用手册分发至多个平台。
  \item 负责积极与学校进行沟通,获得了学校的认可和支持;维护社区,及时响应用户的需求;并撰写了大量文档。目前模板下载量超过4000次。
\end{itemize}


\ResumeItem{区块链在线教学实验平台}
[架构设计、全栈开发]
[2020.12 - 2021.06] 

\begin{itemize}
  \item 设计并实现了基于 Kubernetes 的在线教学平台架构,支持基于用户或者班级的动态容器编排。
  \item 设计并实现一套前后端的平台框架代码(前端采用 Vue.js,后端采用 Spring Boot),提供路由、数据统计、用户信息管理以及实验环境管理能力,使团队内的其他同学只需要专注于开发具体教学实验逻辑。
  \item 基于此平台,实现了基于 Hyperledger Fabric 的交互教学实验,并协助其他同学实现了比特币、以太坊的相关教学实验。
  \item 编写了相关的 CI/CD 以及部署脚本,在多个环境下部署(脚本或 Ansible)并运维此系统。
\end{itemize}

\ResumeItem{电脑诊所线上预约系统}
[架构设计、全栈开发]
[2019.07 - 2019.10] 

\begin{itemize}
  \item 需求源自于我加入的学生组织日常开展的电脑义诊活动。
  \item 后端采用 Django + REST API;前端(包括管理后台和用户应用)采用 Vue.js 基于 Vuetify/VUX 实现 SPA,开发采用 GitLab 及其 CI,运维使用 Docker-compose 组织容器。
  \item 完成需求分析、系统设计和代码实现。与学校积极对接,接入了学校的统一服务平台,向全校师生提供服务。该系统目前仍在投入使用。
\end{itemize}

\end{document}
